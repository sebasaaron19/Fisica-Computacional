\section{Reflexión}
A lo largo de la elaboración de este trabajo me di cuenta de que todo lo que hacemos, incluso respirar, genera una gran cantidad de $CO_2$ para la cual no hay forma de contener o reducir. A diferencia de lo que muchos creen, incluyendo al presidente de los Estados Unidos Donald Trump, el calentamiento global no es ningún mito; se trata de una realidad a la cual hay que enfrentarnos ya si queremos tener un futuro, no es posible que confiemos en películas y libros de ciencia ficción que "predicen" el fin del mundo y no creamos en lo que la ciencia nos dice acerca de este.
Las soluciones están planteadas, depende de nosotros implementarlas o no, si queremos seguir viviendo en este planeta o no. 
Hoy en día existen muchos proyectos que pueden ayudar en gran medida a nuestro planeta como lo son las torres eólicas, los paneles solares, las plantas hidroeléctricas, la obtención de petroleo a partir de basura, un proyecto mexicano, que obtiene petroleo sin la generación de $CO_2$. Otro proyecto mexicano es la torre que absorbe $CO_2$ y genera oxigeno mediante la utilización de alga viva. En Asia existe el proyecto de una torre que limpia el aire de $CO_2$, regresando aire limpio al ambiente mientras que el carbono que obtiene lo comprime y genera diamantes de este, los cuales son vendidos para la construcción de mas torres como esta. Todos y cada un de estos proyectos son una gran opción para limpiar nuestro ambiente, sin embargo la tecnología que se emplea en estos proyectos aun es muy costosa para su producción en masa aparte de que genera una gran disminución de ganancias para la industria petrolera y energética.

Yo creo que hoy ya no es momento de estar pensando en obtener ganancias grandes ni ser el numero uno en ventas alrededor del mundo, si no que debemos buscar la manera de mejorar nuestro ambiente e intentar salvar nuestro planeta, el tiempo esta corriendo y es muy limitado


\section{Referencias Bibliográficas}
National Graographic, "Impacto del cambio climático peor de lo esperado, advierte informe global":
\textcolor{blue}{https://www.nationalgeographic.com/environment/2018/10/ipcc-report-climate-change-impacts-forests-emissions/}

Noticias de la ONU, "Un informe sobre el calentamiento global, una 'llamada de atención' para advertir al jefe de la ONU":
\textcolor{blue}{https://news.un.org/en/story/2018/10/1022492}

World Resources Institue, "8 Things You Need to Know About the IPCC $1.5^\circ$C Report":
\textcolor{blue}{https://www.wri.org/blog/2018/10/8-things-you-need-know-about-ipcc-15-c-report}




 